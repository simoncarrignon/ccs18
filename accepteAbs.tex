\documentclass[]{article}
\usepackage{lmodern}
\usepackage{amssymb,amsmath}
\usepackage{ifxetex,ifluatex}
\usepackage{fixltx2e} % provides \textsubscript
\ifnum 0\ifxetex 1\fi\ifluatex 1\fi=0 % if pdftex
  \usepackage[T1]{fontenc}
  \usepackage[utf8]{inputenc}
\else % if luatex or xelatex
  \ifxetex
    \usepackage{mathspec}
  \else
    \usepackage{fontspec}
  \fi
  \defaultfontfeatures{Ligatures=TeX,Scale=MatchLowercase}
\fi
% use upquote if available, for straight quotes in verbatim environments
\IfFileExists{upquote.sty}{\usepackage{upquote}}{}
% use microtype if available
\IfFileExists{microtype.sty}{%
\usepackage{microtype}
\UseMicrotypeSet[protrusion]{basicmath} % disable protrusion for tt fonts
}{}
\usepackage[unicode=true]{hyperref}
\hypersetup{
            pdfborder={0 0 0},
            breaklinks=true}
\urlstyle{same}  % don't use monospace font for urls
\IfFileExists{parskip.sty}{%
\usepackage{parskip}
}{% else
\setlength{\parindent}{0pt}
\setlength{\parskip}{6pt plus 2pt minus 1pt}
}
\setlength{\emergencystretch}{3em}  % prevent overfull lines
\providecommand{\tightlist}{%
  \setlength{\itemsep}{0pt}\setlength{\parskip}{0pt}}
\setcounter{secnumdepth}{0}
% Redefines (sub)paragraphs to behave more like sections
\ifx\paragraph\undefined\else
\let\oldparagraph\paragraph
\renewcommand{\paragraph}[1]{\oldparagraph{#1}\mbox{}}
\fi
\ifx\subparagraph\undefined\else
\let\oldsubparagraph\subparagraph
\renewcommand{\subparagraph}[1]{\oldsubparagraph{#1}\mbox{}}
\fi

\date{}
\title{Evolution of Cultural Complexity 2018\\--\\Book of Abstracts}
\author{27 September 2018}

\begin{document}
\maketitle

\vspace{1cm}
\begin{center}
    \Large
    {\bf Invited Speakers\\}
    \rule{4cm}{.4pt}
\end{center}


{\bf { \href{../speakers\#ak}{Anne
Kandler} }}

\textbf{Title:} Inferring processes of cultural transmission: the
critical role of rare variants

\textbf{Abstract:} Understanding how social information is used in human
populations is one of the challenges in cultural evolution. Fine-grained
individual-level data, detailing who learns from whom, would be most
suited to answer this question empirically but this kind of data is
difficult to obtain especially in pre-modern contexts. Therefore
inference procedures have often been based on population-level data in
form of frequency distributions of a number of different variants of a
cultural trait at a certain point in time or of time-series that
describe the dynamics of the frequency change of cultural variants over
time, often comprising sparse samples from the whole population. In this
talk we demonstrate that there exist theoretical limits to the accuracy
of the inference of underlying processes of cultural transmission from
aggregated data highlighting the problem of equifinality especially in
situations of sparse data. Crucially we show the importance of rare
variants for inferential questions. The presence, or absence, of rare
variants as well as the spread behaviour of innovations carry a stronger
signature about underlying processes than the dynamic of high-frequency
variants. On the example of the choice of baby names, we illustrate that
the consistency between empirical data, summarized by the so-called
progeny, and hypotheses about cultural evolution such as neutral
evolution or novelty biases depends entirely on the completeness of the
data set considered. Analyses based on only the most popular variants,
as is often the case in studies of cultural evolution, can provide
misleading evidence for underlying processes of cultural transmission.

\rule{4cm}{.4pt}

{\bf { \href{../speakers\#ak}{Peter Turchin} }}

\textbf{Title:} The Evolution of Complex Societies: Old Theories and New Data 

\textbf{Abstract:} Over the past 10,000 years human societies evolved from “simple”—small egalitarian groups, integrated by face-to-face interactions, —to “complex”—huge anonymous societies with great differentials in wealth and power, extensive division of labor, elaborate governance structures, and sophisticated information systems. One aspect of this “major evolutionary transition” that continues to excite intense debate is the origins and evolution of the state—a politically centralized territorial polity with internally specialized administrative organization. Theories proposed by early theorists and contemporary social scientists make different predictions about causal processes driving the rise of state-level social organization. I will use Seshat: Global History Databank to empirically test predictions of several such theories. I will present results of a dynamic regression analysis that estimates how the evolution of specialized governance structures was affected by such factors as social scale (population, territorial expansion), social stratification, provision of public goods, and information systems.

\begin{center}
    \Large
    {\bf Contributed Talks\\}
    \rule{4cm}{.4pt}
\end{center}


{\bf Dries Daems}

\textbf{Title:} Materialising complexity. A conceptual model of material
culture, social complexity and mechanisms of change

\textbf{Abstract:} The genesis of complex societies has captivated
scientific minds across disciplinary boundaries. In archaeology,
trajectories of social complexity were traditionally considered from
reductionist evolutionary perspectives focusing on fixed stages of
societal development. In response, two strands of thought developed: 1)
In the 1980's and 90's, archaeologists started to stress the
multivocality of microscale behavioural complexity in everyday practices
of social life as expressed through the entangled interaction between
people and material objects; 2) Since the turn of the millennium, the
waxing and waning of social complexity has increasingly come to be
considered in light of resilience, sustainability, and transformation in
macroscale dynamics of stability and change in complex societies. The
immense potential of combining these micro- and macroscale approaches,
has so far been insufficiently realized.\\
The present paper aims to bridge this gap by presenting a conceptual
model which integrates material culture -- expressed through social
practices and flows of information -- as micro-level building blocks for
macro-scale dynamics of societal change and stability. To formalise this
model, I focus on three general mechanisms of change -- differentiation,
specialization, and connectivity -- operating within a framework of
complexity as a problem-solving tool. I will look in particular at
developments in five main domains: 1) subsistence and raw material
procurement; 2) technology; 3) inter-group competition; 4)
socio-political structures; and 5) (economic) production. In this
perspective, complex societies develop as people and social groups on
various levels, domains and scales become increasingly interrelated
within nested structures of functional, informational, and
decision-making roles.

\rule{4cm}{.4pt}

{\bf {Thibaud Gruber and Dora Biro}}

\textbf{Title:} Efficiency as a driver of cultural evolution: from birds
to primates

\textbf{Abstract:} While evidence for socially transmitted behaviour in
a variety of species supports claims of cultural variation in wild
animals, cultural evolution in animals itself remains a controversial
topic, because of limited evidence for progression toward more complex
behaviour. Animal ``cultures'' remain largely seen as perpetually
re-invented by each new generation of a given population, with little
progression from one variant to another across generations. We believe
this view results mainly from the theoretical approach applied to
cultural evolution, inspired by modern humans, which tends to blend the
concept of cultural evolution with an increase in cultural complexity,
the ratchet, scaffolded by high-fidelity social learning processes such
as imitation or teaching. While we agree that increase in complexity has
characterized much of human cultural evolution, and possibly some animal
behavioural traits, we believe that complexity may not be a driver per
se of cultural evolution. Rather, both animals and humans select for
greater efficiency, which may in turn select for more complex behaviour
as a by-product. We will analyze examples from the literature and some
of our recent studies in this light: the spread of moss-sponging as an
alternative to leaf-sponging in wild chimpanzees, and the cumulative
learning, across artificial generations, of travel routes in homing
pigeons. We argue that both examples may be considered evidence of
cumulative cultural evolution, which arose through selection for greater
efficiency, rather than complexity. Accordingly, efficiency rather than
complexity may thus be the main driver for cumulative cultural
evolution.

\rule{4cm}{.4pt}


{\bf {Jelena Grujic, \href{http://www.mcdonald.cam.ac.uk/}{Miljana
Radivojevic} and Marko Porcic}}

\textbf{Title:} The concept of archaeological cultures -- an inside from
complex networks approach

\textbf{Abstract:} The concept of archaeological culture is one of the
most challenging yet most enduring concepts in prehistoric archaeology.
Recently we applied an innovative method based on complex networks
analysis to identify community structures in the archaeological record
and investigate pathways to an independent evaluation of archaeological
cultures that produced and traded copper in the Balkans, from c. 6200 to
c. 3200 BC. Used only trace element data of 410 copper-based objects
from 79 archaeological sites as the independent variable for detecting
the most densely interconnected sets of archaeological sites we
uncovered modular structures that exhibit strong spatial and temporal
significance within each observed time slice across c. 3,000 years. Here
we build upon our previous study and apply an improved modelling
approach to empirically determine if traditionally defined
archaeological cultures of the Balkan Neolithic and Chalcolithic
(6200-3200 BC) represent meaningful entities from the perspective of the
most densely connected copper supply networks and if an agreement
between obtained modular structures and archaeological data is
plausible. Furthermore, we improve our previous method by conducting
cluster analysis of the bipartite network instead of its projection, as
in our previous study. Finally, we present a reinforced model of human
interaction and cooperation that can be evaluated independently of
established archaeological systematics, and can find wide application on
any quantitative data from archaeological and historical record.

\rule{4cm}{.4pt}


{\bf {\href{http://hobsonresearch.com/}{Elizabeth Hobson},
\href{http://pure.au.dk/portal/en/danm@econ.au.dk}{Dan Mønster} and
\href{http://tuvalu.santafe.edu/~simon/}{Simon Dedeo}}}

\textbf{Title:} Detecting the Basis of Sociocultural Complexity in
Animals and Humans

\textbf{Abstract:} The extreme social and cultural complexity of human
groups is a fundamental feature that differentiates human sociality from
animals, but despite long-standing interest, the evolution of
sociocultural complexity in both humans and animals is still poorly
understood. Most studies use a bottom-up measure, where social
complexity is contained within the number, type, or strength of pairwise
relationships in groups. However, this perspective loses a lot of the
complexity of sociocultural structures. Rather than using a bottom-up
measure, which focuses exclusively on the structure of local social
interactions, we describe a novel integrated feedback loop as a way to
bridge between local and global properties of sociality, where
individual actions both create the group's social world and can then be
influenced by these social structures. We show how these methods can
lead to new understanding of sociocultural complexity in the context of
within-group conflict. We apply these methods to observational studies
of over 85 species of animals as they choose who to fight with and to
experimental studies of humans as they synthesize social information and
formulate conflict strategies within a networked computer game. This
approach provides new potential for broad comparative analyses to better
understand the evolution of complex sociocultural traits.

\rule{4cm}{.4pt}

{\bf \href{https://francoislafond.info}{Francois Lafond}}

\textbf{Title:} The evolution of classification systems as indicator of
cultural evolution

\textbf{Abstract:} It has long been recognized by anthropologists and
sociologists that classification systems reflect prevalent institutions
and cognitive organizations. In this talk, I will describe my
preliminary attempts at bringing a complex system twist to this strand
of research -- using a data driven approach to understand empirical
patterns, and describing classification systems as stochastically
evolving networks where simple rules of evolution lead to empirically
realistic classification trees. I will present a few case studies of
technological and economic classification systems, and in particular the
US patent classification system, which evolved for almost two centuries
[1], and in which items are reclassified when the classification
system is updated [2]. I will discuss opportunities and challenges
associated with using this data to understand and predict long-run
innovation.\\[2\baselineskip]{[}1{]} Lafond, F. and Kim, D. (2017)
Long-run dynamics of the U.S. patent classification system,
https://arxiv.org/abs/1703.02104\\

{\bf [}2{]} Verendel, V., Lafond, F. and Farmer, J.D. (2018), The origins of
new technological domains, in progress, to be presented at the CCS 2018.

\rule{4cm}{.4pt}

{\bf {\href{https://www.physics.auckland.ac.nz/people/done006}{Dion
O'Neale}, Caleb Gemmell, Thegn Ladefoged, Alex Jorgensen, Hayley Glover,
Christopher Stevenson and Mark McCoy}}

\textbf{Title:} Constructing socio-political networks from obsidian
artefacts in pre-European Aotearoa/New Zealand

\textbf{Abstract:} The Polynesian colonists who settled New Zealand some
700 years ago, brought with them cultural conceptions of chiefdom based
on genealogical affiliation (whakapapa) and territory (mana whenua). It
has been suggested that the initial settlers lived in relatively
autonomous villages, and that over centuries these grew to form
geographically larger social units (hapū), which eventually coalesced
into tribal groups known as iwi. We have used archaeological records to
construct networks of obsidian movements in pre-European Aotearoa New
Zealand, and to investigate factors that may have influenced how iwi
groups gathered resources, be they geographic, economic, or social.\\
We create a bipartite network of obsidian source locations and the
archaelogical study sites where the artefacts were ultimately found.
Analysis of the spatial and temporal aspects of the source-site
bipartite network is used to provide insight into the movement and
interactions of the groups who were collecting, transporting, and using
the obsidian. The bipartite networks allow us to test various hypotheses
that might explain the unique distribution of obsidian throughout the
Northland and Auckland regions of Aotearoa/New Zealand. Using tools such
as similarity measures and community detection we identify those regions
with similar patterns of obsidian sourcing which we use to infer social
networks in pre-European Aotearoa/New Zealand.

\rule{4cm}{.4pt}


{\bf {\href{http://www.bsc.es/romanowska-iza}{Iza Romanowska}, Simon
Carrignon and Tom Brughmans}}

\textbf{Title:} When culture meets economy: modelling cultural
complexity in an economic setting

\textbf{Abstract:} When culture meets economy: modelling cultural
complexity in an economic setting Imagine going to a market to buy a new
plate. The seller offers you a wide selection of locally made or
imported ceramics, some cheaper, some more expensive. But which one to
choose? Here we present a model of economic preference designed to
investigate how simple customer preferences can shape centuries long
term economic and cultural trends. By applying a number of standard
cultural evolution algorithms (conformity bias, prestige bias, neutral
etc.) to a baseline economic model (utility maximisation or `sell high,
buy low') we investigate how cultural behavioural scenarios can lead to
different patterns in economic data. Does a complete dominance of one
type of good signify a strong preference of the buyers or can this
pattern arise from other types of cultural bias? Can a high level of
variability in terms of products be equated with more complex
behavioural patterns? Our goal is to provide a benchmark for a more
informed interpretation of cultural assemblages, such as pottery found
at archaeological sites, and to understand what kind of processes might
have driven the apparent changes in cultural complexity over centuries
long time spans. To showcase the utility of these abstract
cultural/economic models we provide a case study centred on Jerash, a
medium sized Roman town in present-day Jordan, where recent excavations
revealed that the local pottery dominates the archaeological record for
a period of six centuries. The results of our agent-based model indicate
that this pattern could have arisen only within a narrow band of
conditions giving us an unprecedented window into the lives and
decisions of ancient inhabitants of Jerash.

\rule{4cm}{.4pt}

{\bf \href{https://hcommons.org/members/nevrome/}{Clemens Schmid}}

\textbf{Title:} A computational Cultural Transmission model of Bronze
age burial rites in Central, Northern and North-western Europe

\textbf{Abstract:} European Bronze age archaeology traditionally
focusses on two major dimensions to categorise burials -- although
there's an immense variability of attendant phenomena within this
spectrum: Flat graves versus burial mounds and cremation versus
inhumation. These traits are an indispensable ingredient for common
narratives of sociocultural interaction in the Bronze
age.

This complex system of ideological affiliation and
exchange can be described in the terms of Cultural Evolution theory.
Burial rites are extraordinary cultural traits: Following Dunnels$^1$
distinction between function and style based on relevance for selection,
they can be handled as neutral variants. As demonstrated by Neiman$^2$,
drift and intergroup transmission as opposed to natural selection should
therefore be the decisive processes for their expansion. On the other
hand funerals touch upon personal loss and profound religious beliefs:
They are not short-lived fashion and most probably well interlinked with
other -- many archaeologically inaccessible --
traits.

This paper will present the results of my
currently ongoing master thesis. To study the diffusion of burial rites,
I employ the dataset RADON-B$^3$ which contains more then two thousand
Bronze age $^{14}$C dates of graves from Central, Northern and
North-western Europe. Based on this information I construct regional
time series that document how rituals change. For a better understanding
of the real-world phenomena I implement a computational model in R and
C++. It simulates the expansion of ideas in an artificial population
graph and provides an environment to explore the effects of parameters
like group size or the degree of intergroup idea transmission.
\\[2\baselineskip]$^1$
Dunnell, Robert C., ‘Style and Function: A Fundamental Dichotomy’, American Antiquity, 43 (1978), 192–202 https://doi.org/10.2307/279244 \\

$^2$Neiman, Fraser D., ‘Stylistic Variation in Evolutionary Perspective: Inferences from Decorative Diversity and Interassemblage Distance in Illinois Woodland Ceramic Assemblages’, American Antiquity, 60 (1995), 7–36 https://doi.org/10.2307/282074 \\

$^3$
Kneisel, Jutta, Martin Hinz, and Christoph Rinne, ‘Radon-B’, 2013 http://radon-b.ufg.uni-kiel.de. \\


\rule{4cm}{.4pt}


{\bf {Kaarel Sikk, Geoffrey Caruso and Aivar Kriiska}}

\textbf{Title:} Conceptual framework of assessing the influence of
cultural complexity to settlement pattern formation

\textbf{Abstract:} Settlement patterns are one of the main products of
prehistorial archaeological research and are used as spatial projections
of past societies. In current paper we study how geographical locational
data can reveal information about cultural complexity. The formation of
the patterns is influenced by multiple factors from human-environment
interactions to complex processes within society.\\
We analyse the forces behind formation of settlement patterns from an
agent based modelling perspective. For the purpose we construct a
spatial discrete choice model and formulate it using random utility
theory. We argue that agent decisions in the models can be decomposed
into different rulesets. Those rules are mostly determined by attraction
to natural affordances and sociocultural behaviours.\\
Paleoecological and geological data can be used to extract information
about human attraction to natural affordances. Analysing the resulting
empirical data can reveal the significance of environment as determining
settlement choice which we argue is declining with growing cultural
complexity.\\

\rule{4cm}{.4pt}

{\bf Christopher Watts}

\textbf{Title:} Simulating institutional innovation and the collapse of
complex societies

\textbf{Abstract:}
In this presentation we discuss early-stage work on testing, via agent-based simulations,
Joseph Tainter’s (1988) theory of the collapse of complex societies. Tainter argued that
ancient societies solved resourcing problems by becoming more complex, that is, by
increasing in the number and diversity of people, places, goods, practices, hierarchy, roles
and rules. Performing and reproducing the increased complexity, however, costs resources,
especially energy, leading to future problems, requiring further innovation. When the
returns to complexity turn negative, the alternative solution is a dramatic simplification, or
collapse. This theory has since received interest from researchers into the sustainability of
contemporary resource use. To study the complexification-and-collapse phenomena, we
propose simulating agents who use resources, make institutional judgments as part of
governance systems for resource use, and can collectively innovate in their innovations. We
discuss the empirical support for complexity and collapse in the context of ancient societies,
the motivation for a focus on institutions, previous simulation models of innovation,
emergent organisation and resource use from which we may learn, and some of the
technical challenges we anticipate in developing the new model.

\rule{4cm}{.4pt}

%{\bf {Jorge Castillo, Ignacio Toledo and Carlos Rodriguez}}
%
%\textbf{Title:} Evolution of painting content: A data-based analysis
%
%\textbf{Abstract:} Painting is one of the oldest cultural expressions of
%the human species and has evolved according to historical processes.
%However, no quantitative methods have been yet developed in order to
%measure its evolution. In this research, we propose a content-based
%heterogeneity measure to characterize cultural changes in painting. For
%this purpose we combine two computational methods: image content
%analysis and structural topic models, to analyze pictures content in an
%unsupervised way. We apply this method to a randomized sample of\\
%5000 paintings created between 1080 and 2013, Results show the clusters
%of concepts developed in painting, allowing identify the emergence of
%new genres. Also the content heterogeneity measure shows an evolutionary
%dynamic with stasis phases and some gradual and punctuated transitions,
%being consistent with how art historians describe this phenomena. A
%discussion will lead us to how a methodological proposal in
%data-analysis can be useful for extract relevant information in order to
%test cultural evolutionary hypotheses.

\end{document}
